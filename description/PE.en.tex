\gdef\thisproblemauthor{}
\gdef\thisproblemdeveloper{}
\gdef\thisproblemorigin{}
\begin{problem}{奪回作戰}
{standard input}{standard output}
{1 seconds}{512 MB}{}

Sana竟然輕鬆地被其他國家的特務抓走了,這對於太陽國國家形象可是奇恥大辱!因此被任命保護Sana的女僕幹員得竭盡全力阻止這場綁架行動。

\centerline{\includegraphics[scale=0.3]{./pics/E.png}}

女僕幹員身懷各式絕技,當她使出第$K$個絕招時,體內的能量會隨著時間每秒蓄力逐漸增\textbf{遞增},如果完全蓄力時能量值表示為$1/1$,第$i$秒可表示為$a_i/b_i$,對於所有的$i$都滿足:
\begin{itemize}
\item $0\leq a_i \leq b_i$
\item $1\leq b_i \leq K$
\item $gcd(a_i,b_i)=1$,gcd表示兩數的最大公因數
\item $a_i/b_i < a_{i+1}/b_{i+1}$,且兩數之間不存在其他數字滿足上述條件
\end{itemize}

比如說如果女僕幹員使出了第$5$號絕招,那她體內的能量變化依序會是:$0/1, 1/5, 1/4, 1/3, 2/5, 1/2, 3/5, 2/3, 3/4, 4/5, 1/1$。

而女僕幹員蓄力完成之後,釋放的招式強度與能量變化有關,假設能量變化依序是$s_1,s_2,\cdots s_k,s_k=1/1$,若$s_i=a_i/b_i$,那招式強度就等於$b_i/b_{i+1}:1\leq i\leq k-1$的總和!如果你知道女僕幹員要使用的招式,你能計算出女僕幹員的招式強度嗎?

\InputFile

第一行有一個整數$T$,表示接下來有幾筆測資。每一筆測資佔一行,只有一個數字$K$,表示女僕幹員要使用的招式。

\begin{iofmt}
\begin{itemize}
	\item $1\leq T \leq 10000$
	\item $1\leq K \leq 10^6$
	\item 有$10$分的測資$K\leq 10$
	\item 有$20$分的測資$K\leq 20$
	\item 有$40$分的測資$K\leq 1000$
\end{itemize}
\end{iofmt}

\OutputFile

對於每個招式輸出一行,為招式強度的\textbf{最簡分數},若分母為$1$則只需要輸出分子一個數字即可。

\Examples

\begin{example}
\exmpfile{./sample/PE.sample.in}{./sample/PE.sample.out}%
\end{example}

\end{problem}
