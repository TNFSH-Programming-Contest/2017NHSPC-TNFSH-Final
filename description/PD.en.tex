\gdef\thisproblemauthor{}
\gdef\thisproblemdeveloper{}
\gdef\thisproblemorigin{}
\begin{problem}{綁架}
{standard input}{standard output}
{1 seconds}{512 MB}{}

如果能善加利用Sana的神秘力量的話,就可以做出許多當代科技難以達成的超級武器,因此身為邪惡的米國帝國注意到了Sana的能力後,立即派遣出了帝國最精密的幹員:幹員C要綁架Sana,讓她成為米國帝國黑科技的推手!

\centerline{\includegraphics[scale=0.5]{./pics/D.png}}

可是要把具有強大能力的Sana帶走不是一件容易的事情,從Sana出生開始,身上的能力強度無時無刻都在增長,已知出生第$1$秒時,Sana的能力強度等於$1$單位,第$2$秒時等於$1$單位,第$K$秒的能力強度會等於前一秒的能量強度加上$K-2$秒的能量強度,成長幅度十分驚人。而且Sana對外界能量也具有抵抗力,第$i$秒的抵抗力$D_i$,洽等於從出生第$1$秒到第$i$秒這$i$個能力強度的總和,幾乎是無堅不摧的狀態。

由於執行一次捕捉計畫需要十分巨大的代價,因此幹員C需要精算在Sana出生後第$K$秒時,Sana對外界能量的抵抗力是多少,你能幫幹員C計算這個困難的問題嗎?不過這個數字十分的巨大,難以檢驗,因此只須要告訴幹員C數值除以$1000000009$的餘數就行了。


\InputFile

有多筆測資,但是不超過10筆測資。每筆測資占一行,只有一個數字$i$,表示幹員C要計算Sana第$i$秒的抵抗力。

\begin{iofmt}
\begin{itemize}
	\item $1 \leq i \leq 2^{63}-1$
	\item 有$20$分的測資$i\leq 10$
	\item 有$40$分的測資$i\leq 1000$
	\item 有$60$分的測資$i\leq 10^{10}$
\end{itemize}
\end{iofmt}

\OutputFile

對於每一筆測資輸出一行,為抵抗力的數值。

\Examples

\begin{example}
\exmpfile{./sample/PD.sample.in}{./sample/PD.sample.out}%
\end{example}

\end{problem}
