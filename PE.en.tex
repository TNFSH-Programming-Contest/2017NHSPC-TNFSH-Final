% !TEX root = ./problem.en.tex
\gdef\thisproblemauthor{}
\gdef\thisproblemdeveloper{}
\gdef\thisproblemorigin{}
\begin{problem}{模數 Modular}
{standard input}{standard output}
{2 seconds}{512 MB}{}

Jill 最喜歡糖果了!\newline
為此她想開一間自己的糖果工廠來滿足無止盡的糖果欲!\newline
\newline
這間工廠除了生產糖果以外還要兼顧美感和娛樂性,其中包裝糖果是個很重要的工作。\newline
包裝的工作由一個巨大的包裝器負責,這個包裝器由 N 道等待隊列組成,由左至右編號成 0 ~ N-1。包裝器會不定時的接收產出的一些糖果,並將這批糖果塞進其中一個隊列,等待包裝。每次包裝時,機器會選取一段連續的區間 [ L , R ],將 R-L+1 個容量為 v 的包裝運至這些編號的隊列下方,而後區間中每個隊列落下 v 個糖果,進行包裝。要是隊列中沒有 v 個以上的糖果,那這個隊列將不會落下任何糖果 ( 沒裝滿太失禮了!)\newline
\newline
為了確認生產過程的視覺效果,Jill 需要模擬生產行為,其中視覺效果的關鍵就在於隊列中的糖果堆放數量差異。Jill 將會給你一些生產過程的事件,並且不定時詢問你某個區間 [l,r] 內的隊列之中,堆放最多和最少糖果的兩者幾個糖果。\newline
\newline
※每個隊列的容量都很大,不需要考慮滿出來的問題\newline


\InputFile

一個正整數 $N$ 表示隊列數量 \newline
下一行一個正整數 $M$ 表示接下來有 $M$ 個事件或來自 Jill 的詢問 \newline
接下來 $M$ 行每行輸入一個事件或詢問 \newline
每行第一個正整數 $K$ 表示這行是代表什麼事件或詢問 \newline \newline
說明如下:
\begin{itemize}
\item $K=1$ $\Rightarrow$ [ 將糖果塞進隊列 ] \\
            接著會輸入兩個正整數 $x$, $i$ \\
            表示將x個糖果放進編號 $i$ 的隊列
\item $K=2$ $\Rightarrow$ [ 進行包裝 ] \\
            接著輸入三個整數 $L$,$R$,$v$ \\
            表示編號 $L$ 到 $R$ 的隊列要進行容量為 $v$ 的包裝
\item $K=3$ $\Rightarrow$ [ Jill 的詢問 ] \\
            接著輸入兩個正整數 $L$,$R$ \\
            表示 Jill 想知道 編號 $L$ 到 $R$ 之中堆放最多和最少的差異
\end{itemize}

% todo
% \begin{iofmt}
% \begin{itemize}
% 	\item $1 \leq N \leq 10^4$
% 	\item 有 5 分的測試資料 $N \leq 15$
% 	\item 有 15 分的測試資料 $N \leq 100$
% \end{itemize}
% \end{iofmt}

\OutputFile

對於每個詢問,輸出一個正整數,表示區間內最多糖果與最少糖果的差

\Examples

\begin{example}
% \exmpfile{./sample/PC-01.in.txt}{./sample/PC-01.out.txt}%
% \exmpfile{./sample/PC-02.in.txt}{./sample/PC-02.out.txt}%
\end{example}

\end{problem}
