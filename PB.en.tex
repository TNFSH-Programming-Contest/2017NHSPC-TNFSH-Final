% !TEX root = ./problem.en.tex
\gdef\thisproblemauthor{}
\gdef\thisproblemdeveloper{}
\gdef\thisproblemorigin{}
\begin{problem}{魔數 MagicNumber}
{standard input}{standard output}
{2 seconds}{512 MB}{}

多門很喜歡一些很多門的數\newline
一個數字要很多門,需要對於某些正整數$K$,在 $2^K$ 進位下能找到連續的三個1\newline
\newline
例如:\newline
7 在 2 進位 ($2^1$進位) 下是 111\newline
84 在 4 進位($2^2$進位) 下是 1110\newline
\newline
給定一個16進位表示的數字 $N$\newline
請幫多門找出 $N$ 對於哪些 $K$ 會符合很多門的條件\newline

\InputFile

輸入一個 16 進位表示(英文為小寫)的正整數 $N$
\begin{iofmt}
\begin{itemize}
	\item $N \leq 16^{1000000}$
	\item 有 18 分的測試資料 $N \leq 16^5$
	\item 有 41 分的測試資料 $N \leq 16^{1000}$
\end{itemize}
\end{iofmt}

\OutputFile

輸出所有可行的 $K$ \newline
每個佔一行

\Examples

\begin{example}
\exmpfile{./sample/PB-01.in.txt}{./sample/PB-01.out.txt}%
\exmpfile{./sample/PB-02.in.txt}{./sample/PB-02.out.txt}%
\exmpfile{./sample/PB-03.in.txt}{./sample/PB-03.out.txt}%
\exmpfile{./sample/PB-04.in.txt}{./sample/PB-04.out.txt}%
\end{example}
$54_{16} = 84_{10} = 1110_4$ \newline
$\textrm{e}0_{16} = 1110 0000_2$ \newline

\end{problem}
